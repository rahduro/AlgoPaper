\section{Conclusion}
\subsection*{Impact of the Results}
The range minimum query problem has direct impact in several fields related to string pattern matching, text compression, genomic sequence analysis in computational biology and image  processing. It also has applications in OLAP databases where multidimensional range min/max queries are used extensively.\par
Recently Brodal et al.\cite{p11} proposed a space efficient indexing data structure for two-dimensional RMQ problems that improves upon the $O(NlogN)$ bits or $O(N)$ words data structure proposed in this paper. Their paper uses the concept of Fibonacci lattice that has applications in the field of graphics and image processing. The paper by Mitzi et al.\cite{p12} explores the problem of expensive updating of prefix sum cubes for OLAP range queries on flash memory. This is based upon the theoretical research work being done on developing efficient data structures and algorithms that focus on range query computations for a given set of dimensions. Farzan et al. \cite{p14} presented a fully compressed representation of a set of m points on an $n \times n$ grid that uses encoding in higher dimensions. This is again inspired by Yuan's contribution.
\subsection*{Future Work}
Although this paper has shown remarkable improvements from the previous techniques used, still it leaves some scope of improvements which we would like to have in the future. 
\compress
\begin{enumerate}\topsep0pt \itemsep1pt \parskip0pt \parsep0pt
\item From this paper we obtain an upper bound on the linear comparisons required at the preprocessing stage. However this leaves open one question that if $t$ comparisons are allowed at the querying stage then what can be the minimum number of comparisons at the preprocessing stage. A major improvement would be to obtain a tradeoff between the number of comparisons at the querying and preprocessing stage. Demaine et al. has shown some results for Cartesian trees, but an improvement would be extend the results for canonical trees.
\item \textbf{Dynamic Updates: }The array considered for range min queries in this paper, are assumed to be fixed. Any update to any position of the array would invalidate the entire preprocessing data structure. Hence it is again required to preprocess the entire data structure before querying stage. The data structure suggested by Poon \cite{p6}
will be much less impacted by such dynamic updates. Hence in the future, a major improvement would be to have space efficient data structures and algorithms which would enable dynamic updates on canonical trees.
\item This paper has based its solution for range minimum query problem on the RAM model. However we still do not know its impact in the external memory model environment. What remains to be seen is that what will be the cost of implementing macro/mircro data structures and the number of comparisons required for the querying and preprocessing stages.
\end{enumerate}
\par
