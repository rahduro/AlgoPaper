\section{Introduction}
This paper\cite{p10} addresses the problem of answering the range minimum query for higher dimensional arrays. Formally the range minimum query problem can be defined as follows:\par
For a d-dimensional array A of size $N = n_1.n_2...n_k$, where $n_k$ is the length of the $k^{th}$ dimension, a d-dimensional range minimum query problem aks for the minimum element in the query range $q = [a_1, b_1]$ x $[a_2, b_2]$ x..x $[a_d, b_d]$, where $a_k$ and $b_k$ are the upper and lower indices of the $k^{th}$ dimension respectively.
\begin{center}
\begin{math}
RMQ(A, q) = min A[q] = min_{(i_1,....,i_d)\in q}  A(i_1,....,i_d)
\end{math}
\end{center}

This paper gives the first linear time and space preprocessing algorithm with constant query time for d-dimensional arrays where d is assumed to be fixed. In order to achieve this goal, new data structure has been proposed for which the following two conditions must be met:\par
\begin{enumerate}
\item It does not compute all the possible RMQ results at the preprocessing stage. Instead it allows the flexibility of comparisons and generating new results at the querying stage. This greatly reduces the number of types to linear.\par
\item The encoding of the data structure can be computed in linear time.
\end{enumerate}
Their result can be stated as follows:\par
For a d-dimensional array, $\mathcal{O}((2.89)^d.(d+1)!)N)$ comparisons are sufficient to preprocess the input array and the querying stage requires $\mathcal{O}(2^d-1)$ comparisons. Hence this gives a linear time preprocessing and constant time querying for higher dimensional arrays where d is assumed to be fixed. When this scheme is implemented under the RAM model with a fixed dimension, the querying time is increased to $\mathcal{O}(3^d)$, which is still a constant, while the preprocessing time remains the same.\par
The RMQ problem for higher dimensional problems have been studied before and this paper made a major breakthrough by extending the 1D RMQ problem to higher dimensional arrays while preserving the linear time preprocessing and constant time querying features. The intial thought was to compute cartesian tree like structures in linear time which can also capture all possible RMQ results.\par
In fact, it was proved by Demaine, Landau and Weimann\cite{p2} that there is no such possible cartesian tree like structure in 2D case. The main point of difference of their data structure with a cartesian tree is that while the latter requires to get range minimum without any comparison at the querying stage, their data structure allows a constant number of candidates to be compared at the querying stage. 
\section{Related Work}
Here we present some previous works that we presume to be milestones in the field of \emph{RMQ}. \emph{RMQ} was studied intensively in late 1980's and early 1990's for its application in string and pattern matching. Researchers have used techniques to reduce 1D \emph{RMQ} to \emph{LCA} (Least Common Ancestors) problems in linear time for solving LCA. Though 1D RMQ gained attention so early, multi-dimensional RMQ problems were left aside for long.
To realise the contribution and effect of this paper, it is necessary to revisit the timeline of RMQ development. \\

Gabow et al. \cite{p5} (1984) proposed reduction of 1D \emph{RMQ} to \emph{LCA} problems, and solved them with Cartesian trees\cite{p9}. Harel, Tarjan et al. \cite{p4} is the first paper to give linear time preprocessing of array to achieve constant time querying on \emph{LCA}.  The paper extended the results to multidimensional RMQ and their results are not independent of $N$ and $d$. Preprocessing time and space of multidimensional array of size $N$ is $\mathcal{O}(N \log ^{d-1} N)$ which in turn helps in reducing querying time to $\mathcal{O}(\log^{d-1} N)$. \\

Let us briefly discuss the approach given in \cite{p5}. Firstly they proved that in a cartesian tree for any indices $i,j,1\leq i\leq j\leq n$, $min$ \{ $x_k$ $|$ $i \leq k \leq j$\} = $lca(x_i,x_j)$. Hence \emph{RMQ} can be done by answering $lca$ computations. The Cartesian tree is found by constructing sequence $T_i,$ $i = 1,\ldots,n$ where $T_i$ is the Cartesian tree for \{$x_j$ $|$ $j=1,\ldots,i$\}. $T_i$ is derived from $T_{i-1}$ by ascending along the rightmost path of $T_{i-1}$ until $x_k$, the first node larger than $x_i$, is found; the right subtree of $x_k$ is made into the left subtree of $x_i$, and $x_i$ is made the right child of $x_k$. The total time to construct $T_n$ = $T$ is $\mathcal{O}(n)$, since a node leaves the rightmost path immediately after it is traversed.\\