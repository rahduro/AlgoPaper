\section{Notational Preliminaries}
Let's define $A$ as the input array of $d$-dimensions. We assume the length of each dimension denoted by $n_k$ for $1\leq k\leq d$ to be power of $2$. This assumption simplifies the explanation of the algorithms without any loss of generality. And it also would not change any complexity bounds derived. The indices for $k^{th}$ dimension is denoted by $[1,n_k]$. For a $d$-dimensional range $q=[a_1,b_1]\times[a_2,b_2]\times\ldots\times[a_d,b_d]$, $A[q]$ is defined to be the subarray induced by $q$. Now $RMQ(A,q)=\min A[q]$, i.e. the minimum element in the subarray $A[q]$. The indices for $k^{th}$ dimension a subarray are assumed to be $[1,b_k-a_k+1]$. The array entries are assumed to be distinct without any loss of generality, so that for any range there is only $1$ minimum element. But this can be enforced by breaking ties by introducing some consistent ordering (like lexicographic ordering).$POS(A,q)$ is a $d$-dimensional vector denoting index of $RMQ(A,q)$. The size of a range is defined as the number of integer coordinates that are contained in the range and is denoted by $\left\vert q\right\vert$. For example the range $q$ mentioned in the beginning has $(b_1-a_1+1)\times(b_2-_2+1)\times\ldots\times(b_d-a_d+1)$ elements in it where $a_i,b_i$ are all integers. Therefore, if $\left\vert q\right\vert =1$, then $A[q]$ denotes a particular entry instead of a subarray.